\documentclass[oribibl]{llncs}

\usepackage{units}
\usepackage{psfrag} %% psfrac does not work with pdflatex
\usepackage{amssymb}
\usepackage{amsmath}
\usepackage{booktabs}
\usepackage{microtype}
\usepackage{subfigure}
\usepackage{todonotes} %% add [disable] to disable
\usepackage{transparent}
\usepackage{pgfplots}

\usepackage{acronym} %% for abbreviations
\acrodef{dft}[DFT]{density functional theory}
\acrodef{sho}[SHO]{spherical harmonic oscillator}
\acrodef{ho}[HO]{harmonic oscillator}
\acrodef{planewave}[PW]{plane wave}
\acrodef{gridpoint}[GP]{grid point}
\acrodef{paw}[PAW]{Projector Augmented Wave}


\setlength{\tabcolsep}{6pt}

\newcommand{\um}[1]{_{\mathrm{#1}}}
\newcommand{\ttt}[1]{\texttt{#1}}
\newcommand{\lmax}{\ell_{\mathrm{max}}}
\newcommand{\ellm}{L}
\newcommand{\nrn}{n_{\mathrm{r}}}
\newcommand{\ket}[1]{\left| #1 \right\rangle}
\newcommand{\bra}[1]{\left\langle #1 \right|}
\newcommand{\braket}[2]{\left\langle \left. #1 \right| #2 \right\rangle}
\newcommand{\braketop}[3]{\left\langle \left. #1 \right| #2 \left| #3 \right. \right\rangle}

\begin{document}
\pagestyle{plain}

\title       {Spherical Harmonic Oscillator Basis}
\titlerunning{Spherical Harmonic Oscillator Basis}

\author{%
  Paul F.~Baumeister\inst{1} % \and %
}

\institute{%
  J\"{u}lich Supercomputing Centre, Forschungszentrum J\"{u}lich, 52425 J\"{u}lich, Germany
%   \and Institute for Advanced Simulation, Forschungszentrum J\"{u}lich, 52425 J\"{u}lich, Germany
}

\maketitle

% ==============================================================================
\begin{abstract}
\todo[inline]{write abstract}
\end{abstract}
% ==============================================================================



% ==============================================================================
\section{Introduction} \label{sec:intro}
% ==============================================================================
%
Real-space grid based methods for \ac{dft}
have proven to yield good parallelizability and the same level of
accuracy as \ac{planewave} basis sets.
The latter aspect is in particular true for results converged
in terms of the number of \acp{planewave} compared to those converged
w.r.t.~the grid spacing.
However, real-space grid methods cannot be operated at very coarse
grid spacings due to unphysical interferences between the position
of atoms relative to the position of grid points.
This leads to relatively high cost prefactors (for time and space)
of the real-space methods compared to \ac{planewave} basis sets
when we want to do a fast but less accurate calcuation.
Furthermore, the iterative solver schemes applied in real-space
grid formalism often deteriorate in terms of their convergence
velocity when the number of grid points increases.
This is due to the limited width of the stencil compared to the 
global scale of the non-local solutions.
Therefore, convergence acceleration is crutial here.
In many situations, switching to a \ac{planewave} representation
is a viable option for an efficient preconditioner as
\acp{planewave} contain the full non-locality.
Then again, \acp{planewave} destroy the parallelizability
to some extend.
The approach investigated in this project is a small basis of atom-centered
localized orbitals.
This basis is not meant to produce as accurate results as \acp{gridpoint} or \acp{planewave}
or to feature a advantageous convergence w.r.t.~costs
but to be simple and cheap in its construction (low number of control parameters)
and to produce approximately right physics already at small basis sets.
If in addition we can define an efficient transformation between
representations in
the small localized basis set and grid-based representations
this basis can be used to accelerate
the congerence of a real-space grid-based method.

% ==============================================================================
\section{Spherical Harmonic Oscillator} \label{sec:sho}
% ==============================================================================
%
The quantum-mechanical \ac{ho} has the Hamiltonian
\begin{equation}
  \hat H^{[1D]}_{\sigma} = -\frac{\partial_x^2}{2} + \frac{x^2}{2 \sigma^4} \text{.}
  \label{eqn:HO-Hamiltonian}
\end{equation}
Hartree atomic units are used throughout this document.
Here, $\sigma$ is a lengthscale parameter that also fixes the scale of the eigenenergies
\begin{equation}
  E^{[1D]}_{n}(\sigma) = \sigma^{-2} \left( n + \frac 12 \right) \text{.}
  \label{eqn:HO-eigenenergy}
\end{equation}
The \ac{ho} eigenfunctions are
\begin{equation}
  \psi^{[1D]}_{n}(x) = H_n(x/\sigma) \cdot \exp\left( -\frac{x^2}{2 \sigma^2} \right) 
  \label{eqn:HO-eigenfunction}
\end{equation}
with the Hermite polynomials $H_n$.
\todo[inline]{add normalization constants}

The eigenfunctions of the quantum-mechanical 
three-dimensional isotropic harmonic oscillator 
- in the following we will refer to it as \ac{sho} -
can be written as Cartesian product of three
\ac{ho} eigenfunctions
\begin{equation}
  \hat H^{[3D]}_{\sigma} = -\frac{\vec \nabla^2}{2} + \frac{\vec r^2}{2 \sigma^{4}} 
  \label{eqn:SHO-Hamiltonian}
\end{equation}
has the solutions
\begin{equation}
  \psi^{[3D]}_{n_x n_y n_z}(x,y,z) = \psi^{[1D]}_{n_x}(x/\sigma) 
                               \cdot \psi^{[1D]}_{n_y}(y/\sigma) 
                               \cdot \psi^{[1D]}_{n_z}(z/\sigma) 
  \label{eqn:SHO-eigenfunction}
\end{equation}
and the eigenenergies
\begin{equation}
  E^{[3D]}_{n_x n_y n_z}(\sigma) = \sigma^{-2} \left( n_x + n_y + n_z + \frac 32 \right) 
  \label{eqn:SHO-eigenenergy}
\end{equation}

Furthermore, the \ac{sho} can be solved in spherical coordinates 
exploiting its spherical symmetry.
This leads to the quantum numbers $\nrn, \ell$ and $m$
and the eigenstates
\begin{equation}
  \psi_{\nrn \ell m}(r,\vartheta,\varphi) = R_{\nrn \ell}(r/\sigma) 
                               \cdot Y_{\ell m}(\vartheta,\varphi)
  \label{eqn:SHO-eigenfunction-radial}
\end{equation}
with the radial function
\begin{equation}
  R_{\nrn \ell}(r) = r^\ell \cdot L^{(\ell + \frac 12)}_{\nrn}(r^2) \cdot \exp(-\frac{r^2}2)
  \label{eqn:SHO-radial-eigenfunction}
\end{equation}
where $L_n^{(\alpha)}$ stands for the associated Laguerre polynomials.
\todo[inline]{add normalization constants}

In the code, spherical harmonics $Y_{\ell m}$ usually appear in their representation
$X_{\ell \mu}$ where each $\mu$ stands for a real-valued linear combination of $m$ and -$m$.
\todo[inline]{cite Homeier}

\subsection{SHO as a basis}
An infinite set of all \ac{sho} eigenfunctions forms a basis of the 3D function space.
Limiting the set by cut-off energy $E\um{cut} = \sigma^{-2} (\nu\um{max} + \frac 32)$
creates a finite basis.
If we create a union of finite bases centered at each atom $a$ in a \ac{dft} calculation 
we only need to choose a $\sigma$ and $\nu\um{max}$ for each atomic species.
\begin{equation}
  \chi_{a n_x n_y n_z}(x,y,z) = \psi^{[1D]}_{n_x}((x - x_a)/\sigma_a) 
                          \cdot \psi^{[1D]}_{n_y}((y - y_a)/\sigma_a) 
                          \cdot \psi^{[1D]}_{n_z}((z - z_a)/\sigma_a)
  \label{eqn:localized-basis}
\end{equation}
See fig.~\ref{fig:HO-basis-on-2-atoms}.
%
\begin{figure}
  \begin{minipage}[c]{.58\textwidth}
	\includegraphics[width=\textwidth]{fig/HO-basis-on-2-atoms} %%
  \end{minipage}\hfill
  \begin{minipage}[c]{.41\textwidth}
  
%%%% generated by fig/HO-basis-on-2-atoms.F90
%    0.082  -0.183   0.290   1.000  -0.000  -0.000
%   -0.183   0.328  -0.389  -0.000   1.000   0.000
%    0.290  -0.389   0.288  -0.000   0.000   1.000

		\begin{tabular}{r rrr}
		\toprule
				    & $\ket{s_0}$ & $\ket{p_0}$ & $\ket{d_0}$ \\
% 		\midrule
				$\bra{s_1}$  &      0.082 &  -0.183 &  0.290  \\
				$\bra{p_1}$  &     -0.183 &   0.328 & -0.389  \\
				$\bra{d_1}$  &      0.290 &  -0.389 &  0.288  \\
		\bottomrule
		\end{tabular}

  \end{minipage}
  \label{fig:HO-basis-on-2-atoms}
  \caption{
Schematic picture to illuminate the non-orthogonal localized basis in 1D.
On each atomic position \ac{ho} basis functions are centered.
Here, $\sigma$ has been chosen such that $d$-orbitals can form bonds (green lines).
The non-trivial part of the overlap matrix is shown on the right.
  }
\end{figure}
%
%

Then the evaluation of matrix elements works as follows:

\subsubsection{Overlap matrix elements}
Basis functions $\chi$ centered at the same atom $a$ are orthogonal by construction
so we only need to take care of their normalization.
For any pair of atoms $(a,a')$, we can compute the overlap element
\begin{equation}
  \braket{ \chi_{a n_x n_y n_z} }{ \chi_{a' n'_x n'_y n'_z} } = \prod_{i=1}^D
  \int\mathrm d x_i \  \psi^{[1D]}_{n_{x_i}}((x_i - x_{ia})/\sigma_a)
                    \  \psi^{[1D]}_{n'_{x_i}}((x_i - x_{ia'})/\sigma_{a'})
  \label{eqn:overlap-factorized}
\end{equation}
In each of the integrals we can exploit that the product 
of two Gaussians with spread $\sigma_a$ and $\sigma_{a'}$ 
centered at $x_a$ and $x_{a'}$, respectively, 
is again a Gaussian with spread $\sigma_p = \left(\sigma_a^{-2} + \sigma_{a'}^{-2}\right)^{-\frac12}$
centered at $\sigma_p^2 \left( \sigma_a^{-2} x_a + \sigma_{a'}^{-2} x_{a'} \right)$.
The polynomial expressions can be shifted to the new center and multiplied there.
See fig.~\ref{fig:overlap_1D} for an impression of the 1D overlap values 
as a function of distance the centers.
Finally, the integration reduces the the evaluation of the kernel
\begin{equation}
  I_k = \int\limits_{-\infty}^{\infty} \mathrm d x \  \exp(-x^2) \  x^{2k}
  \label{eqn:gauss-integral-kernel}
\end{equation}
which can be evaluated recursively by $I_{k+1} = (k + \frac12)\,I_k$ starting at $I_0 = \sqrt{pi}$.
Mind that no approximations are required here.
%
\begin{figure}
  \begin{minipage}[c]{.990\textwidth}
	\includegraphics[width=\textwidth]{fig/overlap_1D} %%
  \end{minipage}\hfill
  \begin{minipage}[c]{.009\textwidth}
  \end{minipage}
  \label{fig:overlap_1D}
  \caption{
	The overlap integral values of two 1D \ac{ho} eigenfunctions (both with spread $\sigma$)
	as a function of distance between their centers.
	Only an upper triangle of the combinations $(n,n')$ up to $n\um{max}=3$ is shown.
  }
\end{figure}
%
%


\subsubsection{Hamiltonian matrix elements}
The \ac{dft} Hamiltonian consists of a kinetic energy operator,
a local effective potential 
and, if the basis is incapable of
capturing the rapid oscillations close to the nuclei,
a non-local contribution according to the \ac{paw} method.
A similar \ac{paw} contribution would then also appear in the overlap operator.
The kinetic energy consists of a Laplacian so we can boild it down to the evaluation
of $\braketop{ \psi^{[1D]}_{an} }{ \partial^2_x }{ \psi^{[1D]}_{a'n'} }$.
This works just as outlined for the overlap elements
except for the derivative operator modifying the polynomials.

The non-linearity of the exchange-correlation potential in \ac{dft} usually
leads to a representation of the local potential $V(x,y,z)$ on a real-space grid.
We will assume a Cartesian uniform grid here.
Then, viable options would be
\begin{itemize}
%
\item to evaluate $\braketop{ \chi }{ \hat V }{ \chi' }$ numerically.
%
\item for each pair $(a,a')$ which is not too distant
compute an expansion of $V(\vec r)$ into a \ac{sho} basis with spread $\sigma_p$
at the common center and do the product and integral there.
%
\item for each atom $a$ expand $V(\vec r)$ into a suitable \ac{sho} basis. 
Then, we can insert a completeness relation:
\begin{equation}
 \braketop{ \chi_{a\vec n} }{ \hat V }{ \chi_{a'\vec n'} } = \sum_{\vec n''}
 \braketop{ \chi_{a\vec n} }{ \hat V }{ \chi_{a\vec n''} } \braket{ \chi_{a\vec n''} }{ \chi_{a'\vec n'} }
\end{equation}
such that all Gaussian are centered at the position of atom $a$ for the expectation value
and the second integral is the same as for the overlap operator.
Mind that we inserted a complete basis $\chi_{a\vec n''}$. 
This means that the convergence w.r.t.~the cut-off for $\vec n''$ needs to be checked carefully.
Due to different convergence rates, the potential matrix $\hat V$ might not be symmetric
requiring symmetrization according to $\hat V\um{sym} = \frac 12 \left( \hat V + \hat V^\dagger \right)$.
%
\end{itemize}
In the following, we will pursue the latter approach.

Finally, non-local potentials are usually expressed as dyadic operators
$\ket{\tilde p^a_i} D^a_{ij} \bra{\tilde p_j}$ with localized projector functions $\tilde p_i(\vec r)$
centered at the atomic positions.
We can expand the projectors in a suitable \ac{sho} basis and use the same method
as for overlap elements~\cite{BaumeisterTsukamotoPASC19}.

\subsubsection{Expansion of the Potential}
If the local \ac{sho} basis with $\sigma$ and $\nu\um{max}$
is suitable to describe wave functions, 
a suitable \ac{sho} basis for the expansion of $V(\vec r)$ has
spread $\sigma_a/\sqrt{2}$ and $2\nu\um{max}$.
This is due to the representation of densities and potentials
arising from those densities, c.f.~fig.~\ref{fig:plot_parabolas}.
%
\begin{figure}
  \begin{minipage}[c]{.990\textwidth}
	\includegraphics[width=\textwidth]{fig/plot_parabolas} %%
  \end{minipage}\hfill
  \begin{minipage}[c]{.009\textwidth}
  \end{minipage}
  \label{fig:plot_parabolas}
  \caption{
Schematic picture to illuminate the relation between the SHO basis configuration 
used to describe densities and potentials 
compared to the SHO basis for wave functions:
If wave functions are expanded into a Hermite-Gauss basis with spread $\sigma$
up to $n\um{max}$ (here, the $n\um{max} = 5$ function is shown in blue)
the square of such a wave function leads to the function shown in purple:
A Gaussian with spread $\sigma^{[\varrho]}$ times a polynomial with max.~degree $n\um{max}^{[\varrho]}$.
This function can be exactly expanded by a Hermite-Gauss basis if
$\sqrt{2}\,\sigma^{[\varrho]} = \sigma$
and
$n\um{max}^{[\varrho]} = 2n\um{max}$.
The highest Hermite-Gauss function for $n\um{max}^{[\varrho]} = 10$ is shown in red.
The associated parabolas are shown in bold black and bold green.
In the limit of large $n\um{max}$, the classical return radius $R\um{ret}$
of both SHO bases coincides (grey dashed vertical lines).
The associated cut-off energy is $4$ times higher for the density basis.
  }
\end{figure}
%
%

So for the evaluation of the potential matrix elements,
we have to prepare this tensor of coefficients:
\begin{equation}
	P_{nn'n''} = \int \mathrm d x \ H_{n}(x) \exp(-x^2/2) \cdot H_{n''}(x\sqrt{2}) \exp(-x^2) \cdot H_{n'}(x) \exp(-x^2/2)
\end{equation}
where the indices $n$ and $n'$ are associated with the SHO basis for wavefunctions
and $n''$ runs up to $n\um{max}^{[\varrho]}$. 
Obviously, all $P_{nn'n''}$ are zero if $n + n' + n''$ is odd.

\todo[inline]{Would it be better to expand $V$ in polynomials without Gaussian decay?}

% ==============================================================================
\bibliographystyle{splncs03} \bibliography{literature}
% ==============================================================================

\end{document}
